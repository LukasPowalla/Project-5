\documentclass[10pt,a4paper]{article}
\usepackage[utf8]{inputenc}
\usepackage[english]{babel}
\usepackage[T1]{fontenc}
\usepackage{amsmath}
\usepackage{amsfonts}
\usepackage{amssymb}
\usepackage{makeidx}
\usepackage{graphicx}
\usepackage{fourier}
\usepackage{listings}
\usepackage{color}
\usepackage{hyperref}
\usepackage[left=2cm,right=2cm,top=2cm,bottom=2cm]{geometry}
\title{Most important statements of the paper "Cold uniform spherical collapse revisited"}

\lstset{language=C++,
	keywordstyle=\bfseries\color{blue},
	commentstyle=\itshape\color{red},
	stringstyle=\color{green},
	identifierstyle=\bfseries,
	frame=single}
\begin{document}

\maketitle
\newpage

\section{}


1.: minimal radius depending on number of particles $N$.

\begin{equation}
	R_{min} \propto N^{-\frac{1}{3}}
\end{equation}
2.: Fraction of particles with positive energy $f^p$ can be approximated good through

\begin{equation}
	f^p(N) \approx a + b \cdot log(N)
\end{equation}

with $a = 0.048$ and $b = 0.022$, or

\begin{equation}
	f^p(N) = 0.1 \cdot N^{0.1}
\end{equation}
3.: kinetic energy per unit ejected mass $\frac{K^P}{f^p}$ ($K^p$ is the sum over the kinetic energy of all emitted particles)

\begin{equation}
	\frac{K^P}{f^p} \propto N^{\frac{1}{3}}
\end{equation}
4.: Radial density profiles of the clusters in equilibrium $n(r)$ can be approximated by 

\begin{equation}
	n(r) = \frac{n_0}{1+(\frac{r}{r_0})^4}
\end{equation} 
with $r_0 \propto N^{-\frac{1}{3}}$ and $n_0 \propto N^2$.






\end{document}